\chapter*{Editorial}

Every square metre of the Earth's surface has been mapped, either by men on the ground or since the advent of satellites, by cameras orbiting high above Earth's atmosphere. The motivation behind the adventures of this journal rests on the search for places impossible to sense remotely, for places untrodden and unseen.

This report focuses on the the story of exploration under Migovec as seen and experienced by individuals involved. The aim of this publication is to relate the achievements in a chronological framework, to give a glimpse of what it felt like for the explorers and provide a basis for further exploration. It therefore consists of a mix of reports, photographs, maps and logbook entries and follows the adventures of ICCC and JSPDT cavers over the years 2013-2017.

I must first thank all contributors to this publication: authors, photographers, cartoonists, songwriters... reviewing this source material has been a joy and perhaps more often than I care to admit, an excuse to delve back into the magical world of Migovec. Flawed as it is, I hope this report will be of some help for any future explorers cutting their teeth under the Hollow Mountain. 

It is also my sincere hope that the many friendships forged during the exploration of Sistem Migovec remain strong. Let the privileged relationship between the Imperial College Caving Club and Jamarska Sekcija PD Tolmin continue to flourish, for as long as true  alpine exploration takes place under Migovec.

\name{Tanguy Racine}


Zgodba o Sistemu Migovec – najdaljši jami v Sloveniji se nadaljuje. Po povezavi sistema Vrtnarija s sistemom Migovec se raziskovanja med leti 2013 – 2015 nadaljujejo v jami Vrtnarija. Konec leta 2015 pa ponoven velik uspeh, saj se najdaljšem sistemu pridruži še zadnji večji sistem na Migovcu in sicer sistem Primadona. Celoten Sistem Migovec je torej sestavljen iz treh večjih sistemov oziroma kar osmih jamskih vhodov. Odgovor na vprašanje, kaj bo cilj odprave za leto 2016, je bil na dlani – jama Primadona, za katero vemo, da še ni v celoti raziskana. V letih 2016 ter 2017 se torej  raziskovanja skoncentrirajo v Primadoni in jo v dveh letih podaljšamo, za kar 3.5 km. Dna še nismo dosegli…
Upam, da bo ta tretja publikacija o “Votli gori” vsaj malce približala obseg raziskovanj tega mogočnega sistema. 
ICCC in JSPDT sta lahko več kot ponosna na doseženo in naj bo to skupno sodelovanje v zgled, da se vse da če je volja, sodelovanje ter seveda vztrajnost. 
Ta zgodba še zdaleč ni končana. Veliko je še neodkritega; še vedno nismo presegli magičnih -1000 m, pa izvedli večje jamske potope. Slediti vodi, ter ne nazadnje kaj vse nam sama geologija pove o samem nastanku.  
Upam, da se bodo prijateljstva nadaljevala, novi Mig jamarji opogumili ter raziskovanja nadaljevala. 

\name{Jana \v{C}arga}

\mydelimiter

This book is divided in three parts. The introduction consists of an overview of the history of exploration and cursory glance at the geography of \passage{Tolminski Migovec}. Follows the main body of text, selected stories from expedition cavers in chronological order, by no means a definitive account of all the excursions under the mountain. It would also be a mistake to believe that the recounted stories all follow a logical chain of events; indeed, some are interesting diversions and tangents from what we could term the main flow of exploration, some are physical but not necessarily narrative dead ends. Some are fruitless endeavours or with the benefit of hindsight futile pursuits only taken up in the heat of exploration. Yet in the midst of this flurry of apparently uncoordinated outings, there will be (more often than not) tales of huge success, some anticipated, others coming out of left field. Overall, these stories span the years 2013-2017, which saw systematic summer expeditions by ICCC and continuous JSPDT action. A third part provides a more detailed description of possible trips within the mountain, augmented with larger scale surveys: they are intended for expedition cavers craving to see more of this nearly 40km alpine system. 

\mydelimiter

\begin{quote}
It's been a long way, but we're here.
 
\raggedleft\normalsize\sffamily\textbf{Alan Shepard} \par\end{quote}