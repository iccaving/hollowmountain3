\chapter{Migovec - an introduction}
\begin{marginfigure}
\checkoddpage \ifoddpage \forcerectofloat \else \forceversofloat \fi
\centering
\vspace{50pt}
 \frame{\includegraphics[width=\linewidth]{"images/maps-of-mig/migovec-alongside-andy".jpg}} 
 \caption{The \protect\passage{Migovec Plateau} and a panorama to the south over the \protect\passage{Dinarides} \pic{Andy Jurd}}
 \label{dinarides}
\end{marginfigure}

\section{The Hollow Mountain}
The silhouette of \passage{Tolminski Migovec} is like an old friend to the inhabitants of nearby villages and alps. The sheer cliffs endure the passing of the seasons, while all around, the alpine meadows and pine forests grow boldly green or retreat and sometimes hide under a thick blanket of winter's snows. Its rough crevices, imposing limestone buttresses and bare scree cones impart an austere look on the unmistakable southern precipice, the face \passage{Tolminski Migovec} shows to the casual walkers.

This mountain of \passage{Tolmin} is reared up against a limestone ridge higher still, as if a promise of a larger barren wilderness, towering at 1862\,m elevation above the \passage{Tolminka} valley. \passage{Mig} lies at the western edge of the \passage{Triglavski Narodni Park} (\passage{Triglav National Park}). 


 \begin{pagefigure}
 \checkoddpage \ifoddpage \forcerectofloat \else \forceversofloat \fi
\centering
  \frame{\includegraphics[width=\textwidth]{"images/maps-of-mig/jana_carga_migoec_in_winter".jpg}}
  \label{winter panorama migovec}
  \caption{\protect\passage[cave]{Tolminski Migovec} in Winter, as seen from a popular paraglider spot looking north \pic{Jana Čarga}}
 \end{pagefigure}

The ascension to the summit, a three to four hour walk from the city of Tolmin rewards the curious walker with gorgeous views of the limestone landscapes of the \passage{Julian Alps}. To the south-west, spanning the bay of \passage{Trieste}, the unbroken flatness of the Italian plain scintillates with a myriad of urban lights after dark. South is a world of rolling hills and low plateaus merging into the sky in a blue haze; the \passage{Ba\v{c}a} and \passage{Idrijca} valleys cutting deep furrows in the otherwise gentle scenery. 

The east is barred from view by a line of jagged high peaks, clawing at the sky with bare grey knuckles, the first prominent mountains of the \passage{Julian Alps}. Beyond, another country almost: water drains to form the \passage\protect{Sava} river, collecting in the glacial lake \passage[lake]{Bohinj}, then flows past \passage{Ljubljana}, the capital, towards the \passage{Black Sea}. The majority of the \passage{Triglav National Park} lies beyond the NW-SE oriented ridge, with the \passage{Komna} plateau separating the massif of Tolminski Migovec from the walls of \passage{Triglav}, not 15\,km distant.

 The valley of the \passage[river]{Tolminka} lies to the west of the \passage{Tolminski Migovec} plateau. Resurging through marshland and boulders at the foot of \passage{Osojnica}, the river flows within a deep, glacial valley, bounded on either side by steep cliffs. 

The \passage{Migovec} massif is made up of nearly a kilometer thick stack of well bedded Triassic limestones. During the building of the Alps, these were emplaced on top of younger Jurassic and Cretaceous rocks, which are predominantly made up of limestone alternating with mudstones, and therefore less prone to hosting extensive cave systems.

As a result of the geology, the surface of the undulating, 1x2km \passage{Plateau} is karstified; it is riddled in places with potholes as much as 30m deep, contains `staircase' karst and a vast underground network of caves, the \passage{Migovec System}. A karst landscape is the unmistakable sign that of all possible erosional processes, dissolution of rock is predominant.

The main resurgences associated with a karst system \sidenote{a karst system encompasses all interconnected voids within the limestone whereas a cave system refers to the fraction that is enterable by man} are the aforementioned Tolminka springs to the west, the \passage{Zadla\v{z}\v{c}ica} to the south east and the \passage{Savica} to the north.

\section{Local geography}
 For cave specific research, we refer the reader to the index of colloquial names, located at the end of the book. In the body of text, the named passages are \smallcaps{highlighted thus}.

\subsection{The M-series} 
The M-series as the caves found mostly on the Migovec Plateau and were first explored during the 1974-1994 period. This area encompasses the undulating surface of the Migovec Plateau, as far north as the start of the rise towards \protect\passage{Tolminski Kuk}, and as far south as the main \protect\passage{Limestone Pavement}, a significant depression between \protect\passage{Tolminski Migovec} and the camping spots ( \fref{map:map overlay}).


\paragraph{Overview} Exploration of \passage{Sistem Migovec} by the \textit{Jamarska Sekcija Planinskega Društva Tolmin - JSPDT}, hereafter referred to as the JSPDT, began in earnest in 1974, under the impulse from longstanding members Zoran Lesjak, Brane Bratuž, Stanko Breška and Fischione Alfonz \citep{hm1}. They started exploring, digging and bolting down the main shakeholes of Area M (for Migovec), naming and numbering their most promising leads at the time (\passage{M1} to \passage{M17}). More M number caves were added during the 1994 \textit{Imperial College Caving Club} (ICCC) expedition (M18-M24). Below is a short discussion on the most significant of the M-series caves (see \citet{hm1} for complete details). 

\begin{marginfigure}
\checkoddpage \ifoddpage \forcerectofloat \else \forceversofloat \fi
\centering
 \frame{\includegraphics[width=\linewidth]{"images/maps-of-mig/m2_entrance".jpg}} 
 \caption{The snow plug entrance of M2 is the highest of the eight ways into \protect\passage{Sistem Migovec} \pic{Tanguy Racine}}
 \label{surfaceM2}
\end{marginfigure}

\paragraph{M2} Early on, \passage{M2} or \passage{Kavka Jama} (Jackdaw Cave) was explored to a depth of -350m, and comprised a 120m unbroken and impressive shaft named \passage[pitch]{Silos}. The twin entrances to the cave lay to the east of the Plateau: a sizeable shaft plugged with snow letting light into the main entrance chamber connected to a smaller, free-climbable rift entrance. The termination of M2 at the time was an artificially enlarged squeeze at the end of a draughting, narrow rift with clear depth potential.

\paragraph{M16} In 1982, and after significant digging effort, \passage{M16} was broken into. A 100m to the SW of \passage{M2}, and a stone's throw from the massive, but unfortunately choked M1 gaping entrance, a small, choss filled tube led to the start of a long pitch series, extending down to -547m. The jewel of the cave which, unusually at time ranked as one of the deepest in Slovenia, was the final chamber \passage{Galaktika}, the largest underground volume yet under \passage{Tolminski Migovec}. 

\begin{marginfigure}
\checkoddpage \ifoddpage \forcerectofloat \else \forceversofloat \fi
\centering
 \frame{\includegraphics[width=\linewidth]{"images/maps-of-mig/prima-ent".jpg}} 
 \caption{The large (by \protect\passage{Migovec} standards) entrance of \protect\passage{Primadona}, found off the west cliff of the \protect\passage{Plateau} whose exploration is dealt with in the 2016 and 2017 exploration entries \pic{Rhys Tyers}}
 \label{surfaceprima}
\end{marginfigure}

\paragraph{M18} This area of shakeholes yielded many other caves in the mid 1990s when the Imperial College Caving Club started to systematically explore the mountain. \passage{M18} or \passage[|see{Torn T-Shirt}]{Jama Strgane Srajce} cave, with its entrance less than 10m away from \passage{M2} developed into a sharp canyon until the key \passage{NCB} breakthrough, which enabled two major connections to be forged between \passage{M18}, \passage{M2} and \passage{M16}. Together, these formed \passage{Sistem Migovec} otherwise known as the \passage{Old System}. Through the \passage{M16} entrance, exploration focussed on the deep end at first, with separate sumps encountered near $-969m$ depth and shifted towards the shallower pitch series, with new passage adding up to over $11km$ of cave, most of it vertical.

As shown on the \passage{Area M} (\vref{map:map overlay}), many other blowing holes, potential cave entrance and caverns of lesser significance (as yet) were pushed, named and surveyed over the years. Early ICCC expeditions (1994-1995) continued in the tradition of the M-series, with cave like \passage[|see{B9}]{Jackie's Blower} (\passage{M21}) \passage{Venus Cave} (\passage{M22}), Gulliver's Kipper and \passage[|see{M24}]{PF10} (\passage{M24}, 119\,m deep). \fref[margin]{sec:early history}

\subsection{East of the Plateau} 

\paragraph{Vrtnarija}
Dropping further down the eastern glacial cirque, to the NE of the aforementioned entrances lies \passage{Vrtnarija} also known as  \passage[|see{Vrtnarija}]{Gardeners' World} cave. Though spotted in 1996, the key breakthrough occurred in the year 2000, when a deep pitch series quickly dropped to -350m and was left ongoing. The following year ended on the discovery a long horizontal gallery at -550\,m, which provided the ideal campsite for the next decade of joint cave exploration between ICCC and the JSPDT. \fref[margin]{sec:early vrtnarija}. There is now more than 10km of horizontal cave development at depths >500m, extending as far north as \passage{Tolminksi Kuk} and south as \passage{Planina Kal}. 

\paragraph{The S- series}  This collection of caves lies down in   the \passage{Gardeners' World} valley. In 2013 the numbers 5-7 were added to the list, but it is S1 which has received the most attention so far. Sporadic exploration over the years (up until 2017) met an ascending boulder choke obstacle issuing a very cold draught, which due to the pushing front's proximity with \passage{M16}, was attributed to the same air as flows through the end \passage{Hotline} gallery to the North West.

\begin{marginfigure}
\checkoddpage \ifoddpage \forcerectofloat \else \forceversofloat \fi
\centering
 \frame{\includegraphics[width=\linewidth]{"images/maps-of-mig/M16_entrance".jpg}} 
 \caption{Due to its ease of access \protect\passage{M16} is the chosen entrance for visits to the \protect\passage{Old System} \pic Rhys Tyers}
 \label{surfaceM16}
\end{marginfigure}

\subsection{West of the Plateau} 

\paragraph{Primadona} At the opposite edge of the \passage{Plateau}, and accessed via either a 120\,m cliff abseil or a steep scree climb, the large entrance of \passage{Primadona} was the gateway to another significant system. The cave filled a blank area of mountain far to the west of \passage{Sistem Migovec} (\fref[margin]{sec:early primadona}).

\paragraph{Monatip} \passage{Monatip} was connected to \passage{Primadona} early on. The first is located 100m north of \passage{Primadona} along the cliffside, begins as a near horizontal crawl and breaks into a short series of up- and down pitches heading southeast, before two routes split off. At \passage{Planika Chamber}, a $\approx30$\,m climb leads into the high level series that eventually connected with \passage{Sistem Migovec} near \passage{NCB} in 2015\fref[margin]{sec:twelve_battles}. The other way is down a steep rift breaking into \passage{Alkatraz} chamber and further down into the water chamber near \passage{Sejna Soba}.

\paragraph{Other cliff entrances} Along the nearly 2\,km long stretch of cliffs which characterise the western side of the \passage{Plateau}, several more caves have been discovered e.g. \passage{Ubend571}, which connects into \passage{Primadona} and whose entrance is 70\,m higher along the abseil route or \passage{Gondolin}, located 100\,m north of \passage{Monatip}. \passage{Planika Jama} lies almost directly above the entrance to \passage{Monatip} and the far passages are choked in ice and rubble.

\begin{pagemap}
 \checkoddpage \ifoddpage \forcerectofloat \else \forceversofloat \fi
\centering
  \includegraphics[width=\textwidth]{"images/maps-of-mig/system_overlay".png}
  \caption{Cave passage and topography of Tolminski Migovec, Slovenian National Grid ESPG 3794}
   \label{map:map overlay}
 \end{pagemap}