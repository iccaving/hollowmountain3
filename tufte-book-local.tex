\RequirePackage{etoolbox}
\makeatletter
\newif\if@tufte@margtab\@tufte@margtabfalse
\AtBeginEnvironment{margintable}{\@tufte@margtabtrue}
\AtEndEnvironment{margintable}{\@tufte@margtabfalse}

\newcommand{\classiccaptionstyle}{%
    \long\def\@caption##1[##2]##3{%
        \par
        \addcontentsline{\csname ext@##1\endcsname}{##1}%
        {\protect\numberline{\csname the##1\endcsname}{\ignorespaces ##2}}%
        \begingroup
        \@parboxrestore
        \if@minipage
        \@setminipage
        \fi
        \normalsize
        \@makecaption{\csname fnum@##1\endcsname}{\ignorespaces ##3}\par
        \endgroup}
    \long\def\@makecaption##1##2{%
        \vskip\abovecaptionskip
        \sbox\@tempboxa{\@tufte@caption@font##1: ##2}%
        \ifdim \wd\@tempboxa >\hsize
        \@tufte@caption@font\if@tufte@margtab\@tufte@caption@justification\fi##1: ##2\par
        \else
        \global \@minipagefalse
        \hb@xt@\hsize{\hfil\box\@tempboxa\hfil}%
        \fi
        \vskip\belowcaptionskip}
       \setcaptionfont{\normalfont}
    \let\caption\@tufte@orig@caption%
    \let\label\@tufte@orig@label}
\makeatother

\usepackage{newfloat}
\DeclareFloatingEnvironment[
    fileext=los,
    listname={List of Surveys},
    name=Survey,
    placement=tbhp,
    within=none,
]{survey}

\newenvironment{pagesurvey}{% This enables the full width captions.
    \protect\begin{survey*}[b]
    \classiccaptionstyle
  }{\protect\end{survey*}}

\DeclareFloatingEnvironment[
    fileext=lom,
    listname={List of Maps},
    name=Map,
    placement=tbhp,
    within=none,
]{map}
\newenvironment{pagemap}{% This enables the full width captions.
    \protect\begin{map*}[b]
    \classiccaptionstyle
  }{\protect\end{map*}}




%____________________ MORE FORMATTING OPTIONS


%indexing the passage names and highlighting them in some way the new command is simply \passage{}
%If using passage in the caption environment - \protect\passage{} will work.
%THis also takes an optional argument \passage[detail]{name} if you want to refer to e.g. Primadona abseil, instead of cave, or entrance in the index. 
%Also, you can use \passage[|see{other}]{name} if you really want something specific like Mig see --Migovec for instance




\newcommand{\name}[1]{\vskip1ex\nopagebreak{\raggedleft\large\sffamily\textbf{#1}\par \vskip1ex} \index[aut]{#1|BH} \rmfamily}

\newcommand{\figref}[1]{(Figure \protect\vref{#1})}
\newcommand{\mapref}[1]{(Map \protect\vref{#1})}
\newcommand{\survref}[1]{(Survey \protect\vref{#1})}

\newcommand{\figpageref}[1]{(Figure \protect\ref{#1} on page \protect\pageref{1})}

\definecolor{thirdblue}{cmyk}{0.97, 0.28, 0, 0.71}


%A handy little command which takes inline text as argument and copies it in the margin as a snippet. Useful to highlight stuff and fill margins

\newcommand{\bignote}[1]{\marginnote{\sffamily\normalsize{\textmd{\textit{\flushleft{\textcolor{thirdblue}{...#1...}}}}}%
\rmfamily}#1}

%Use mininame for authors in margin only tikzboxes, otherwise linespacing was changed by the larger font names.
\newcommand{\mininame}[1]{\vskip2ex\nopagebreak{\raggedleft\footnotesize\sffamily\textbf{#1}\par} \index[aut]{#1|BH} \rmfamily}

\renewcommand{\emph}[1]{\textit{#1}}

%If caption runs at bottom of page - use pagefigure environment instead of figure* !
\newenvironment{pagefigure}{% This enables the full width captions.
    \protect\begin{figure*}[b]
    \classiccaptionstyle
  }{\protect\end{figure*}}

%_____________________Definition of the color box environment for the main title pages...
\usepackage[most]{tcolorbox}

\tcbset{    frame code={}
            center title,
            left=10pt,
            right=10pt,
            top=-50pt,
            bottom=10pt,
            colback=white,
            colframe=white,
        width=\dimexpr\textwidth\relax,
            enlarge left by=0mm,
            boxsep=5pt,
            arc=0pt,outer arc=0pt,
            opacityback=0.7
        }
        

%__________________Definition of the Tikz environments for side stories and fancy tables
\usepackage{tikz}
\usetikzlibrary{shapes,snakes}

\tikzstyle{recipe} = [   draw=thirdblue!80, 
                    fill=thirdblue!20, 
                    thick,
                    rectangle, 
                    rounded corners, 
                    inner sep=5pt, 
                    inner ysep=10pt
                    ]
\tikzstyle{recipe-title} = [  font=\sffamily\large, 
                    fill=thirdblue,
                    text=white
                    ]
\tikzstyle{recipe-logo} = [  font=\sffamily\large, 
                    fill=thirdblue,
                    rectangle,
                    rounded corners,
                    draw = thirdblue
                    ]

\tikzstyle{logbook-logo} = [  font=\sffamily\large, 
                    fill=bettergrey,
                    rectangle,
                    rounded corners,
                    draw = bettergrey
                    ]


\newcommand{\ladle}{\protect\includegraphics[height = 4ex]{images/little_insets/ladle.png}}
\newcommand{\logbook}{\protect\includegraphics[height = 4ex]{images/little_insets/logbook.png}}
\newcommand{\explo}{\protect\includegraphics[height = 4ex]{images/little_insets/helmet.png}}
\newcommand{\calc}{\protect\includegraphics[height = 4ex]{images/little_insets/calc.png}}
\newcommand{\dig}{\protect\includegraphics[height = 4ex]{images/little_insets/dig.png}}
\newcommand{\note}{\raisebox{-1ex}{\protect\includegraphics[height = 5ex]{images/little_insets/notes.png}} \qquad}
\newcommand{\enote}{\qquad \raisebox{-1ex}{\protect\includegraphics[height = 5ex]{images/little_insets/notes2.png}}}


\newcommand{\pic}[1]{\raisebox{-0.2em}{\protect\includegraphics[height = 2ex]{images/little_insets/pic.png}}~\hskip0.15em \emph{#1}\relax}

\newcommand{\pen}[1]{\raisebox{-0.2em}{\protect\includegraphics[height = 2ex]{images/little_insets/pen.png}}~\emph{#1}\relax}
\newcommand{\recipecorner}[2]{
                        \begin{marginfigure}
                        \begin{tikzpicture}
                        \node [recipe] (box){%
                            \begin{minipage}{0.9\textwidth}
                            \vspace{5pt}
                            \sffamily\textcolor{thirdblue!}{#2}
                            \end{minipage}
                        };
                        \node[recipe-title, right=30pt] at (box.north west) {#1};
                        \node[recipe-logo, right=5pt] at (box.north west) {\ladle}; 
                        \end{tikzpicture}
                        \end{marginfigure}
}



\tikzstyle{name-dest} = [   draw=bettergrey!80, 
                    fill=bettergrey!20, 
                    thick,
                    rectangle, 
                    rounded corners, 
                    inner sep=5pt, 
                    inner ysep=10pt
                    ]
                    
\tikzstyle{fancytitle} =     [  font=\sffamily\large, 
                    fill=bettergrey,
                    text=white
                    ]

\tikzstyle{table} = [       draw={thirdblue!80},
                    fill={thirdblue!20},
                    thick,
                    rectangle, 
                    rounded corners, 
                    inner sep=5pt, 
                    inner ysep=10pt
                    ]
                    
\tikzstyle{tabletitle} = [  font=\sffamily\large,
                    fill={thirdblue},
                    text=white
                    ]
\newcommand{\numbertable}[1]{
                        \begin{figure*}[h!]
                        \begin{tikzpicture}
                        \node [table] (box){%
                            \begin{minipage}{\textwidth}
                            \centering
                            \vspace{5pt}
                            

                        #1
                            \end{minipage}
                        };
                        \node[tabletitle, right=30pt] at (box.north west) {Number Crunching};
                        \node[recipe-logo, right=5pt] at (box.north west) {\calc}; 
                        \end{tikzpicture}
                        \end{figure*}
    
}                    
\newcommand{\margininbox}[3]{
                        \begin{marginfigure}
                        \begin{tikzpicture}
                        \node [name-dest] (box){%
                            \begin{minipage}{0.95\textwidth}
                            \vspace{5pt}
                            \normalsize{#2}
                            \end{minipage}
                        };
                        \node[fancytitle, right=30pt] at (box.north west) {#1};
                        \node[logbook-logo, right = 5pt] at(box.north west) {\protect#3};
                        \end{tikzpicture}
                        \end{marginfigure}
}



\newcommand{\fullwidthbox}[2]{
   \begin{figure*}[t!]        
        \begin{tikzpicture}
            \node [name-dest] (box){%
                \begin{minipage}{\textwidth}
                \vspace{5pt}
                    \begin{multicols}{2}
                    #2
                  \end{multicols}
                \end{minipage}
            };
            \node[fancytitle, right=30pt] at (box.north west) {#1};
            \node[logbook-logo, right = 5pt] at(box.north west) {\logbook};
        \end{tikzpicture}
    \end{figure*}
}

%__________________End of the definition of: Tikz environments for side stories and fancy tables    

%____________________THE TWO PAGE BACKGROUND MACRO (from:https://tex.stackexchange.com/questions/23860/how-to-include-a-picture-over-two-pages-left-part-on-left-side-right-on-right).
\usepackage{adjustbox}
\usepackage{afterpage}
\usepackage{placeins}


\setcounter{totalnumber}{1}
\setcounter{topnumber}{1}
\setcounter{bottomnumber}{1}
\renewcommand{\topfraction}{.99}
\renewcommand{\bottomfraction}{.99}
\renewcommand{\textfraction}{.01}
\newcommand*\cleartoleftpage{%
  \clearpage
  \ifodd\value{page}\hbox{}\newpage\fi
}

\fancypagestyle{endchapter}{%
  \fancyhf{}% Clear header/footer
  \fancyfoot[OR]{\textbf{\thepage}}%
  \fancyfoot[EL]{\textbf{\thepage}}%
  \renewcommand{\headrulewidth}{0pt}%
}

%changing the spacing of itemize (also ...)
\usepackage{enumitem}
\newcommand{\localtextbulletone}{\textcolor{bettergrey}{\raisebox{.45ex}{\rule{.6ex}{.6ex}}}}
\renewcommand{\labelitemi}{\localtextbulletone}
\newenvironment{citemize}{\begin{itemize}[itemsep=0.25ex,leftmargin=0.5cm, rightmargin=0.5cm, label=\localtextbulletone]}{\end{itemize}}
